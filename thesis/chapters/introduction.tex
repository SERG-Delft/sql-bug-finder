\chapter{Introduction}
\label{chapter:introduction}

The Structured Query Language, also known as SQL, is a domain specific programming language used for managing and interacting with relational database management systems. It was also one of the first languages to utilize the relational model presented by \citet{P000} in his paper and later on became both an ANSI and ISO standard, making it the most widely used database language, with more than 70\% of developers using SQL as shown by a recent study \cite{P999}. 

What makes it even more impressive, is the fact that SQL is not only used in IT but also in various other industries, such as banking, accounting, aviation or commerce. This makes this programming language one of the most widely spread and at the same time it means that there are a lot of people, with different skill levels and backgrounds, writing and using SQL queries. Depending on the understanding level of SQL, some users might directly use queries from forums or other questions and answers websites, such as StackOverflow, which has inherent risks if these queries contain bugs. For this reason, it is important to therefore have tools which can assist developers in detecting not only syntactic errors in SQL queries, but also semantic ones, just as modern integrated development environments suggest code reformatting for various other programming languages. 

Currently, the tools which exist for SQL are mostly focused on detecting only syntactic errors in queries that developers write. While this is still an important aspect, queries should also be checked for semantic issues, since these types of problems can have a huge impact on the system's performance, as we will later explain in this study. Furthermore, most of the existing tools require the complete database schema in order to analyse queries and detect potential semantic errors. Another important point is that most of the current tools are not able to detect issues in dynamically generated SQL queries either, which is very limited since applications might create queries at runtime depending on various input as described by \citet{P007}.

In the background study that we carried out for this thesis we also identified the lack of open source tools focused on detecting semantic bugs in SQL queries. The lack of such tooling, in turn, makes it harder for developers to identify these issues. As pointed out by \citet{P008} in their study on semantic bugs in student written queries, these types of issues are the ones requiring more knowledge and a deeper understanding of the underlying SQL principles in order to solve. Furthermore, these bugs are not addressed as long as the query contains syntax errors, however, most of the time even after the query formulation is correct, developers tend to overlook the final step of checking for potential semantic bugs. Therefore, having automated tools for checking this would offer tremendous help in solving this problem.

The goal of this thesis is two-fold. First, we propose a set of 25 validated heuristics for detecting the most common semantic bugs encountered in SQL queries, based on evidence from previous research. We then implement a tool, using our proposed heuristics, for detecting semantic bugs in SQL queries and measure the accuracy of the tool by performing an extensive manual analysis on a large collection of queries. Second, we explore the prevalence of these semantic bugs in two different SQL datasets, one provided by \citet{P011} containing queries from three different open source projects tracked on GitHub and the other, which we specifically built for this study, containing queries extracted from StackOverflow posts. To the best of our knowledge, this is the largest and most complete dataset of SQL queries collected from StackOverflow, containing not only queries but also various other additional metadata information, which could be used for other interesting research in the future as well.

Our rule based static analysis tool is able to detect semantic bugs in SQL queries with a 97\% accuracy. Furthermore, we observed that out of all the 191,994 collected queries, 36,818 contain at least one semantic bug, which means that 19.17\% of queries contained some semantic problem in their formulation. Also quite interesting, we show that more complex queries, in terms of number of joins, predicates and functions used, tend to suffer from more semantic bugs, an interesting finding which could be used in the future as a metric for early prediction as to whether a query might contain semantic bugs or not.

Our thesis leads to the following four contributions:
\begin{itemize}
    \item A set of validated heuristics to detect semantic bugs in SQL queries together with a prototype implementation of a tool which implements the proposed heuristics and detects semantic bugs in SQL queries.
    \item An empirical study on the prevalence of semantic bugs in two SQL datasets.
    \item A dataset of 172,000 queries extracted from StackOverflow together with additional metadata information.
    \item A replication package containing the queries and other scripts used in our evaluation process that can help with reproducing and improving our results.
\end{itemize}